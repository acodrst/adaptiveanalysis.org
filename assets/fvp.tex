\newcommand\twodigits[1]{%
   \ifnum#1<10 0#1\else #1\fi
  }

  \usepackage{fancyhdr}
  \usepackage{svg}
  \setsvg{inkscape={"/opt/inkscape"}}
  \svgsetup{inkscapelatex=false}
  \fancypagestyle{plain}{
    \fancyhf{}
    \renewcommand{\headrulewidth}{0pt}
    \pagenumbering{arabic}
    \fancyfoot[R]{\small \textit{\thepage}} 
  }

  \pagestyle{plain}
  \usepackage{lipsum}
  \usepackage{datetime}
  \newdateformat{monthdayyeardate}{%
  \THEYEAR-\twodigits{\THEMONTH}-\twodigits{\THEDAY}}
  \usepackage{fontspec}
  \setmainfont{Arial}
  \renewcommand{\abstractname}{\normalsize Overview}
  \definecolor{left}{HTML}{0077BB}
  \definecolor{middle}{HTML}{CC3311}
  \definecolor{teal}{HTML}{009988}
  \definecolor{orange}{HTML}{EE7733}
  \definecolor{blue}{HTML}{0077BB}
  \definecolor{magenta}{HTML}{EE3377}
  \usepackage{xcolor}
  \usepackage{placeins}
  \usepackage{afterpage}
  \usepackage{fvextra}
  \usepackage{float}
  \usepackage[font={it,bf},labelfont={it,bf}]{caption}
  \usepackage{wrapfig}
  \fancyfootoffset{.20in}
  \DefineVerbatimEnvironment{Highlighting}{Verbatim}{breaklines,commandchars=\\\{\}}
  \usepackage{titling}
  \posttitle{\vspace{-70pt}}
  \setlength{\droptitle}{-50pt}
  \usepackage{titlesec}
  \titleformat*{\section}{\centering\Large\bfseries}
  \titleformat*{\subsection}{\normalsize\bfseries}
  \titleformat*{\subsubsection}{\itshape\normalsize\bfseries}
  %\titlespacing*{⟨command ⟩}{⟨left⟩}{⟨before-sep⟩}{⟨after-sep⟩}[⟨right-sep⟩]
  \titlespacing*{\section}{0pt}{20pt}{20pt}[0pt]
  \titlespacing*{\subsection}{-12pt}{8pt}{1pt}[0pt]
  \titlespacing*{\subsubsection}{0pt}{8pt}{1pt}[0pt]
  \posttitle{\par\end{center}\vspace{-70pt}}
  \directlua{luaotfload.add_fallback
   ("emojifallback",
    {
      "NotoColorEmoji.ttf:mode=harf"
    }
   )}
  \setmainfont[RawFeature={fallback=emojifallback}]{Arial}
  \usepackage{tikz}
  \usetikzlibrary{quotes}
  \usetikzlibrary{shapes}
  \usetikzlibrary {arrows.meta} 
  \usetikzlibrary{graphdrawing}
  \usetikzlibrary{graphs}
  \usegdlibrary{trees}
  \usegdlibrary{circular}
  \usegdlibrary{force} 
  \tikzstyle{arrow}=[line width=1pt, ->, >=latex ]
  \tikzstyle{vertex}=[draw,circle,minimum size=14pt]
  \newcommand{\ada}{\href{https://adaptiveanalysis.org}{\hspace{12pt}\begin{tikzpicture}[transform canvas={scale=.14,xshift=-90pt,yshift=-13pt}]\draw[step=1,gray,line width=.5pt] grid (3,3);\draw[line width=10pt,blue,cap=round](.05,2.95) -- (.7,1.7);\draw [line width=10pt,blue,cap=round] (.7,1.7) -- (1.3,2.2);\draw [line width=10pt,blue,cap=round] (1.3,2.2) -- (2.95,.05);\end{tikzpicture}\hspace{2px}}}
  \newcommand{\tsa}{\href{https://doi.org/10.5281/zenodo.7826793}{\textbf{\textcolor{left}3\kern-.3ex\raisebox{-.48ex}{\textcolor{middle}{S}}\kern-.15ex\textcolor{teal}{A}}}\hspace{2px}} 
  \newcommand{\trs}{\href{https://doi.org/10.5281/zenodo.13684896}{\hspace{16pt}\begin{tikzpicture}[simple necklace layout,transform canvas={scale=.4,rotate=-60,xshift=-22pt,yshift=2pt}]\node(1)[vertex,orange,fill=orange]{};\node(0)[vertex,teal,fill=teal]{};\node(2)[vertex,magenta,fill=magenta]{};\path(0)edge[arrow](1);\path(1)edge[arrow](2);\path (2)edge[arrow](0);\end{tikzpicture}}\hspace{2px}}


\clearpage

\section{Introduction}\label{introduction}

My two previous papers Triple System Analysis ( \tsa) and Adaptive
Analysis ( \trs) explain how to use multi-level knowledge graphs for
system analysis (\citeproc{ref-h_triple_2023}{H. 2023})
(\citeproc{ref-h_adaptive_2024}{H. 2024}). A Flow Visualization
Practionary ( \ada) uses the combined material/data flow model from
\trs\hspace{-3pt}, and simplifies the symbols. The reader will find it
helpful to review \tsa and \trs.

\begin{wrapfigure}[19]{r}{0pt}

\centering \includesvg[scale=.80]{images/Top.svg}

\caption{Top}\label{fig:top}

\end{wrapfigure}

\afterpage{\clearpage}

\subsection{Human Cognition First}\label{human-cognition-first}

We tend to work with systems backwards. We look at the exhaust data from
systems and hope to understand our direction, when we should really be
focusing on where we are, where we want to go, and what dangers lie on
our route before looking at the currents propelling our boat. Our
systems should conform to our needs, not the needs of a provider,
framework or existing systems. There can be some savings in the
short-term by going with the flow and purchasing the dominant service;
however, when rapid change in requirements and features are needed to
adapt to new situations, the technical debt accumulated by not leading
with human cognition increases the risk of capsizing in the rapids. To
get our bearings, humans can consider roughly 3 classes of objects
related in one dimension, which can be seen as players, tools, and teams
towards a common goal
(\citeproc{ref-tomasello_understanding_2005}{Tomasello et al. 2005}). We
have limits on how much information we can consider in real-time to make
decisions (\citeproc{ref-zheng_unbearable_2025}{Zheng and Meister
2025}). What form of knowledge works best for the thin layer of
communication that comprises our conscious mind
(\citeproc{ref-murphy_propofol_2011}{Murphy et al. 2011})
(\citeproc{ref-noauthor_decoding_nodate}{{``Decoding the Void''} n.d.})?
Semiotics are cognitive shortcuts that can help. I use icons for \tsa,
\trs, and \ada, rather than titles, to make it clear that I mean the
idea of the entire paper. I use other conventions in the model that help
the reader understand complex systems without dense dialog. Charles
Peirce developed more sophisticated versions of these ideas, and the
title of this paper is an homage to Michael K. Bergman, a follower of
his (\citeproc{ref-bergman_knowledge_2018}{Bergman 2018}). I have had
professional success using knowledge graphs and semiotics in the form of
Gane and Sarson knowledge graphs (\citeproc{ref-h_triple_2023}{H. 2023})
(\citeproc{ref-gane_structured_1977}{Gane and Sarson 1977}). I've spent
much time since then trying to understand why it worked so well and
developing tools, constraints, and methods that helped with the
challenges. Fig.~\ref{fig:top} Shows the set of symbols used in my
combined material and data flow model. The rounded blue boxes are
transformations of data or materials. The teal boxes are agents that are
the sources or sinks of data or materials. The reddish-brown boxes store
data or materials at rest. Each symbol is a node that is connected with
other nodes, and is called a graph. Besides color and node shape, dotted
lines within the node represent data. Solid lines represent materials.
As I explained in \trs, data flow diagrams are behind agents that
operate transforms. This is why I think it is OK to mix the nodes, as
most of the function is behind the screens, the black box of the device
or report that assists the transform. Magenta dots in the corner of a
transform/process node mean you can zoom in to it by clicking. An orange
dot means you can hover for notes and narrative. A blue dot in the lower
right corner means there is a connection to the associated full data
flow.

\subsection{Third Kiss of the Pig}\label{third-kiss-of-the-pig}

This is my third paper. My dad would say it is my ``third kiss of the
pig'', meaning that this is my last chance at getting the prize. Since
I'm immersed in the idea of triples, calling this my last paper seems
appropriate. There should be three. Also, for health reasons, I need to
back off a bit from my pace. I've been working on these ideas every
waking moment since May, 2019, with the rest of my life shoehorned in. I
need to reverse that. I still feel very strongly that this is what I can
add, something that fits within a mature understanding of progress
(\citeproc{ref-project_development_2024}{Project 2024}); however, I need
to take a more balanced approach to my life going forward.

I spent some time this morning considering the format and my toolchain.
The PDF format is useful, as I can upload it and people can view without
additional software. Even if I just add on to the bottom for each
article, no big deal. The PDF is still available, as is the Markdown.
The document is Pandoc friendly, as it is created with Pandoc, so people
can export to whatever format they like. This is a practionary. It does
not delve in to the ideas of \trs or \tsa. I think this will work just
fine.

\clearpage

\section{Practionary}\label{practionary}

\begin{wrapfigure}[80]{r}{0pt}

\centering \includesvg[scale=4.3]{images/toptext.svg}

\caption{GS}\label{fig:gs}

\end{wrapfigure}

\subsection{Graphs}\label{graphs}

\subsubsection{Creating a Graph}\label{creating-a-graph}

In \tsa I wrote about the whiteboard technique to gather information
collaboratively. I also wrote about how these ideas can be thought of as
mind mapping, and even gave an example of how to export a mind map
directly to triples. \trs introduced graph stack format. Let's use that
to create the graph in Fig.~\ref{fig:top} .

\lipsum

\setstretch{.5}
\vspace{10pt}

\phantomsection\label{lst:mod}%
\begin{Shaded}
\begin{Highlighting}[numbers=left,,]

\end{Highlighting}
\end{Shaded}

\setstretch{1}

\clearpage

\section{References}\label{references}

\phantomsection\label{refs}
\begin{CSLReferences}{1}{0}
\bibitem[\citeproctext]{ref-bergman_knowledge_2018}
Bergman, Michael. 2018. {``A Knowledge Representation Practionary.
{AI}3:::adaptive Information.''} 2018.
\url{https://www.mkbergman.com/a-knowledge-representation-practionary/}.

\bibitem[\citeproctext]{ref-noauthor_decoding_nodate}
{``Decoding the Void.''} n.d. Accessed February 5, 2025.
\url{https://radiolab.org/podcast/anesthesia}.

\bibitem[\citeproctext]{ref-gane_structured_1977}
Gane, Chris, and Trish Sarson. 1977. \emph{Structured Systems Analysis:
Tools and Techniques}. {McDonnell} Douglas Systems Integration Company.

\bibitem[\citeproctext]{ref-h_triple_2023}
H., Scott. 2023. {``Triple System Analysis,''} May.
\url{https://doi.org/10.5281/ZENODO.7826793}.

\bibitem[\citeproctext]{ref-h_adaptive_2024}
---------. 2024. {``Adaptive Analysis,''} August.
\url{https://doi.org/10.5281/ZENODO.13684896}.

\bibitem[\citeproctext]{ref-murphy_propofol_2011}
Murphy, Michael, Marie-Aurélie Bruno, Brady A. Riedner, Pierre Boveroux,
Quentin Noirhomme, Eric C. Landsness, Jean-Francois Brichant, et al.
2011. {``Propofol Anesthesia and Sleep: A High-Density {EEG} Study.''}
\emph{Sleep} 34 (3): 283--91.
\url{https://doi.org/10.1093/sleep/34.3.283}.

\bibitem[\citeproctext]{ref-project_development_2024}
Project, The Consilience. 2024. {``Development in Progress. The
Consilience Project.''} July 16, 2024.
\url{https://consilienceproject.org/development-in-progress/}.

\bibitem[\citeproctext]{ref-tomasello_understanding_2005}
Tomasello, Michael, Malinda Carpenter, Josep Call, Tanya Behne, and
Henrike Moll. 2005. {``Understanding and Sharing Intentions: The Origins
of Cultural Cognition.''} \emph{Behavioral and Brain Sciences} 28 (5):
675--91. \url{https://doi.org/10.1017/S0140525X05000129}.

\bibitem[\citeproctext]{ref-zheng_unbearable_2025}
Zheng, Jieyu, and Markus Meister. 2025. {``The Unbearable Slowness of
Being: Why Do We Live at 10 Bits/s?''} \emph{Neuron} 113 (2): 192--204.
\url{https://doi.org/10.1016/j.neuron.2024.11.008}.

\end{CSLReferences}

\newpage

\hfill\break

\newpage
